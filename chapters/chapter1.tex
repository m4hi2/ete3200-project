\chapter{Introduction}
\label{chap:Introduction}
Humans by nature are curious and comfort seeking. We have created technologies to have more control over our lives and 
make it more easier. Modern technologies have given us the power to do things that was unthinkable just a few years ago. 
Our home is the safest place to us and we want to keep it protected at any cost. Having 24 hours a day and 7 days a week 
monitoring on our home keeps our mind free from the tension of a burglar entering our home. 

This project aims to have constant monitoring of a home or facility, logging who is entering and leaving at what times
and have intrusion alert when someone tries to break in from windows and such. We'll be introducing `Two Factor Authentication'
to our home doors and have full control over the whole system from any where of the world over the internet. 

\section{Motivation}
Being worried when away from our beloved home/facility is not a new phenomena. We're always very anxious about our 
precious belongings whenever they leave our line of sight. This prevents us form concentrating on the present. While intrusion
detection systems are readily available on the market, they are built on top of propitiator technologies and should not be 
trusted with personal data. Also, they run the risk of begin out of service due to their reliance on the major cloud services 
providers. 

\section{Objectives}
\begin{enumerate}
    \item To entry log people entering and leaving
    \item To make sure having awareness of someone breaking in 
    \item To have a second factor of authentication
    \item To avoid being vulnerable by loosing a key
\end{enumerate}

\section{Report Outline}
In \autoref{chap:Introduction} the importance of this project along with the motivation behind this project with the objectives has 
has been discussed.

In \autoref{chap:Background} we discuss about different components and a high level overview of the project. The high level overview
includes functional block diagram and system data flow with decision making. 

In \autoref{chap:Design} we outline the design process and decision making for this project. What parts are used 
including their usage can be found here.

In \autoref{chap:Conclusions} we conclude the project, summarizing it and discussing it's limitations and future 
progress scopes. 