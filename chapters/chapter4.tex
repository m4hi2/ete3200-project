\chapter{Conclusions}
\label{chap:Conclusions}
\begin{flushleft}
    This is the final chapter of this report and it summarizes the key features, limitations of the project and suggests
    a future direction for further improvement.
\end{flushleft}

\section{Summary}
    \begin{flushleft}
        This project implements 2FA on smart home/facilities. It uses a Facial Recognition System along with a 
        traditional Keypad for pass code entry. It also uses wireless intrusion detection sensors to prevent intrusions
        going unnoticed. The information from this project can be accessed from anywhere of the world.
    \end{flushleft}

\section{Limitations}
    \begin{flushleft}
        This project uses a single camera with no depth sensor. There is a potential of fooling the system with a 
        photograph of the authorized person. Even though this could not be reproducible in our testing but it is a 
        limitation of the technology used. 
        
        The processor on the main unit Raspberry Pi is not very fast. Hence if more than 10 face is available in front 
        of the camera it struggles to recognize the correct face.
    \end{flushleft}

\section{Direction For Future Work}
    \subsection{Camera}
        \begin{flushleft}
            The problem of fooling the system with a photograph can be solved by using a stereo camera or by adding a 
            depth camera with the system. The face encoding data then can not be used with normal photographs, it has 
            to be collected with a similar depth/stereo camera. 
        \end{flushleft}

    \subsection{Hardware}
        \begin{flushleft}
            The Raspberry Pi 3B used in this project is a very old device. The newer Raspberry Pi 4B is a more capable
            device. This can solve this issue. Also, a full server grade hardware can easily solve this too while
            adding headroom for future expansion. 
        \end{flushleft}